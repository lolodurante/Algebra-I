\documentclass{article}

% Language setting
\usepackage[spanish]{babel}

% Set page size and margins
% Replace `letterpaper' with `a4paper' for UK/EU standard size
\usepackage[letterpaper,top=2cm,bottom=2cm,left=3cm,right=3cm,marginparwidth=1.75cm]{geometry}
\usepackage{multicol}

% Useful packages
\usepackage{amsmath}
\usepackage{amssymb}
\usepackage{graphicx}
\usepackage[colorlinks=true, allcolors=blue]{hyperref}
\usepackage{enumitem}

\title{Guía Álgebra - Práctica 1}
\author{Lorenzo Durante}

\begin{document}
\maketitle

\section{Introducción}
{Esta es la resolución de la primer guía de ejercicios de Álgebra 1 para Ciencias de la Computación en la UBA.}

\section{Conjuntos}
\subsection{Dado el conjunto \texorpdfstring{$A = \{ 1, 2, 3\}$}{A = {1, 2, 3}}, determinar cuáles de las siguientes afirmaciones son verdaderas.}

\begin{enumerate}[label=\roman*)]
    \item $1 \in A$
    \item $\{1\} \subseteq A$
    \item $\{2,1\} \subseteq A$
    \item $\{1,3\} \in A$
    \item $\{2\} \in A$
\end{enumerate}
\textbf{Resolución}

\begin{enumerate}[label=\roman*)]
    \item $1 \in A$: el número $1$ es un elemento que pertenece al conjunto $A$.
    
    \item $\{1\} \subseteq A$: el conjunto $\{1\}$ está contenido en $A$, ya que todos sus elementos pertenecen a $A$.
    
    \item $\{2,1\} \subseteq A$: el conjunto $\{2,1\}$ es un subconjunto de $A$, pues tanto $1$ como $2$ pertenecen a $A$.
    
    \item $\{1,3\} \notin A$: el conjunto $\{1,3\}$ no es un elemento de $A$, es decir, $A$ no contiene a $\{1,3\}$ como uno de sus elementos.
    
    \item $\{2\} \notin A$: el elemento $\{2\}$ no pertenece al conjunto $A$. 
\end{enumerate}

\subsection{Dado el conjunto \texorpdfstring{$A = \{1, 2, \{3\}, \{1,2\}\}$}{A = {1, 2, {3}, {1,2}}}, determinar cuáles de las siguientes afirmaciones son verdaderas.}

\begin{multicols}{2}
\begin{enumerate}[label=\roman*)]
    \item $3 \in A$
    \item $\{3\} \subseteq A$
    \item $\{3\} \in A$
    \item $\{\{3\}\} \subseteq A$
    \item $\{1,2\} \in A$
    \item $\{1,2\} \subseteq A$
    \item $\{\{1,2\}\} \subseteq A$
    \item $\{\{1,2\}, 3\} \subseteq A$
    \item $\emptyset \in A$
    \item $\emptyset \subseteq A$
    \item $A \in A$
    \item $A \subseteq A$
\end{enumerate}
\end{multicols}

\textbf{Resolución}
\begin{enumerate}[label=\roman*)]
    \item $3 \notin A$: es falso ya que el número $3$ no es un elemento del conjunto $A$.
    \item $\{3\} \not\subseteq A$: es falso ya que el elemento $3$ no está en $A$.
    \item $\{3\} \in A$: es verdadero porque el elemento $\{3\}$ pertenece al conjunto $A$.
    \item $\{\{3\}\} \subseteq A$: es verdadero ya que el único elemento de este conjunto es $\{3\}$ y este pertenece a $A$. 
    \item $\{1,2\} \in A$: verdadero, ya que $\{1,2\}$ pertenece a $A$.
    \item $\{1,2\} \subseteq A$: verdadero porque $1$ y $2$ pertenecen a $A$. 
    \item $\{\{1,2\}\} \subseteq A$: verdadero porque $\{1,2\}$ pertenece a $A$. 
    \item $\{\{1,2\},3\} \not\subseteq A$: falso, ya que $3$ no pertenece a $A$. 
    \item $\emptyset \in A$: falso, ya que $\emptyset$ no está como elemento dentro de $A$.
    \item $\emptyset \subseteq A$: verdadero, ya que el conjunto vacío es subconjunto de todos los conjuntos. 
    \item $A \in A$: falso, ya que $A$ no es un elemento de sí mismo.
    \item $A \subseteq A$: verdadero, ya que todo conjunto es subconjunto de sí mismo.
\end{enumerate}

\section*{3. Determinar si $A \subseteq B$ en cada uno de los siguientes casos.}

\begin{enumerate}[label=\roman*)]
    \item $A = \{1,2,3\}, \quad B = \{5,4,3,2,1\}$
    \item $A = \{1,2,3\}, \quad B = \{1,2,\{3\},-3\}$
    \item $A = \{x \in \mathbb{R} \mid 2 < |x| < 3\}, \quad B = \{x \in \mathbb{R} \mid x^2 < 3\}$
    \item $A = \{\emptyset\}, \quad B = \emptyset$
\end{enumerate}

\textbf{Resolución}
\begin{enumerate}[label=\roman*)]
    \item $A \subseteq B$
    \item $A \not\subseteq B$
    \item $A \not\subseteq B$
    \begin{align*}
        A &= [-3,-2] \cup (2,3) \\
        B &= (-\sqrt{3}, \sqrt{3}) \\[6pt]
        &\text{Entonces, } A \nsubseteq B \\
        &\text{Por ejemplo, } -2.5 \in A \;\text{ pero }\; -2.5 \notin B.
    \end{align*}
    \item $A \not\subseteq B$
\end{enumerate}

\subsection{Dados los subconjuntos}
\begin{align*}
A &= \{1,-2,7,3\}, \\
B &= \{1,\{3\},10\}, \\
C &= \{-2,\{1,2,3\},3\}
\end{align*}
del conjunto referencial
\[
V = \{1,\{3\},-2,7,10,\{1,2,3\},3\},
\]
hallar
\begin{enumerate}[label=\roman*)]
  \item $A \cap (B \triangle C)$
  \item $(A \cap B)\triangle (A \cap C)$
  \item $A^{c}\cap B^{c}\cap C^{c}$
\end{enumerate}

\textbf{Resolución}
\begin{enumerate}[label=\roman*)]
    \item $A \cap (B \triangle C) = \{1, -2, 3\}$
    
    Pienso el ejercicio por partes: primero analizo $B \triangle C$. La diferencia simétrica contiene lo que está en uno u otro, pero no en ambos.
    \begin{align*}
        B &= \{1,\{3\}, 10 \} \\
        C &= \{-2,\{1,2,3\},3\} \\
        B \triangle C &= \{1, \{3\}, 10, -2, \{1,2,3\}, 3\}
    \end{align*}
    Sea $D = B \triangle C$, evaluemos ahora la intersección entre $A$ y $D$.
    \begin{align*}
        A &= \{1,-2,7,3\}, \\
        D &= \{1, \{3\}, 10, -2, \{1,2,3\}, 3\}\\
        A \cap D &= \{1, -2, 3\}
    \end{align*}
    Entonces, el resultado de la intersección es $\{1, -2, 3\}$.
    
    \item $(A \cap B) \triangle (A \cap C) =  \{1, -2, 3\} $ \\[6pt]
    Primero analizo la primer intersección:
    \begin{align*}
        A &= \{1,-2,7,3\}\\
        B &= \{1, \{3\}, 10\}\\
        A \cap B &= \{1\}
    \end{align*}
    Ahora analizo la segunda intersección:
    \begin{align*}
        A &= \{1, -2, 7, 3\}\\
        C &= \{-2, \{1,2,3\}, 3\}\\
        A \cap C &= \{-2, 3\}
    \end{align*}
    Ahora podemos calcular la diferencia simétrica:
    \begin{align*}
        A \cap B &= \{1\} \\
        A \cap C &= \{-2, 3\} \\
        (A \cap B) \triangle (A \cap C) &= \{1, -2, 3\}
    \end{align*}
    
    \item $A^{c} \cap B^{c} \cap C^{c} = \emptyset$ \\[6pt]
    El complemento se toma respecto al conjunto referencial $V$. 
    
    \textbf{Primer complemento:}
    \begin{align*}
        A &= \{1,-2,7,3\}\\
        V &= \{1, \{3\}, -2, 7, 10, \{1,2,3\},3\}\\
        A^{c} &= \{\{3\}, 10, \{1,2,3\}\} 
    \end{align*}
    
    \textbf{Segundo complemento:}
    \begin{align*}
        B &= \{1,\{3\}, 10\}\\
        V &= \{1, \{3\}, -2, 7, 10, \{1,2,3\},3\}\\
        B^{c} &= \{-2, 7, \{1,2,3\}, 3\}
    \end{align*}
    
    \textbf{Tercer complemento:}
    \begin{align*}
        C &= \{-2, \{1,2,3\}, 3\}\\
        V &= \{1, \{3\}, -2, 7, 10, \{1,2,3\}, 3\}\\
        C^{c} &= \{1, \{3\}, 7, 10\}
    \end{align*}

    \textbf{Intersección final:}
    \begin{align*}
        A^{c} &= \{\{3\}, 10, \{1,2,3\}\}\\
        B^{c} &= \{-2, 7, \{1,2,3\}, 3\}\\
        C^{c} &= \{1, \{3\}, 7, 10\}\\
        A^{c} \cap B^{c} \cap C^{c} &= \emptyset
    \end{align*}
\end{enumerate}

\end{document}
