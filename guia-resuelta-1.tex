\documentclass{article}

% Language setting
\usepackage[spanish]{babel}

% Set page size and margins
% Replace `letterpaper' with `a4paper' for UK/EU standard size
\usepackage[letterpaper,top=2cm,bottom=2cm,left=3cm,right=3cm,marginparwidth=1.75cm]{geometry}
\usepackage{multicol}

% Useful packages
\usepackage{amsmath}
\usepackage{amssymb}
\usepackage{graphicx}
\usepackage[colorlinks=true, allcolors=blue]{hyperref}
\usepackage{enumitem}

\title{Guia Álgebra - Práctica 1}
\author{Lorenzo Durante}

\begin{document}
\maketitle

\section{Introducción}
{Esta es la resolución de la primer guía de ejercicios de Álgebra 1 para Ciencias de la Computación en la UBA.}

\section{Conjuntos}
\subsection{Dado el conjunto \texorpdfstring{$A = \{ 1, 2, 3\}$}{A = {1, 2, 3}}, determinar cuáles de las siguientes afirmaciones son verdaderas.}

\begin{enumerate}[label=\roman*)]
    \item $1 \in \mathcal{A}$
    \item $\{1\} \subseteq \mathcal{A}$
    \item $\{2,1\} \subseteq \mathcal{A}$
    \item $\{1,3\} \in \mathcal{A}$
    \item $\{2\} \in \mathcal{A}$
\end{enumerate}
\textbf{Resolución}
\begin{enumerate}[label=\roman*)]
    \item $1 \in A$: el número $1$ es un elemento que pertenece al conjunto $A$.
    
    \item $\{1\} \subseteq A$: el conjunto $\{1\}$ está contenido en $A$, ya que todos sus elementos pertenecen a $A$.
    
    \item $\{2,1\} \subseteq A$: el conjunto $\{2,1\}$ es un subconjunto de $A$, pues tanto $1$ como $2$ pertenecen a $A$.
    
    \item $\{1,3\} \notin A$: el conjunto $\{1,3\}$ no es un elemento de $A$, es decir, $A$ no contiene a $\{1,3\}$ como uno de sus elementos.
    \item $\{2\} \notin$ A: el elemento $\{2\}$ no pertenece al conjunto A. 
\end{enumerate}

\subsection{Dado el conjunto \texorpdfstring{$A = \{1, 2, \{3\}, \{1,2\}\}$}{A = {1, 2, {3}, {1,2}}}, determinar cuáles de las siguientes afirmaciones son verdaderas.}

\begin{multicols}{2}
\begin{enumerate}[label=\roman*)]
    \item $3 \in A$
    \item $\{3\} \subseteq A$
    \item $\{3\} \in A$
    \item $\{\{3\}\} \subseteq A$
    \item $\{1,2\} \in A$
    \item $\{1,2\} \subseteq A$
    \item $\{\{1,2\}\} \subseteq A$
    \item $\{\{1,2\}, 3\} \subseteq A$
    \item $\emptyset \in A$
    \item $\emptyset \subseteq A$
    \item $A \in A$
    \item $A \subseteq A$
\end{enumerate}
\end{multicols}

\textbf{Resolución}
\begin{enumerate}[label=\roman*)]
    \item $3 \notin A$: es falso ya que el número $3$ no es un elemento del conjunto $A$.
    \item $\{3\} \not\subseteq A$: es falso ya que no todos los elementos estan contenidos por el conjunto A. En este caso, el elemento 3 no existe en el conjunto A.
    \item $\{3\} \in A$: es verdadero porque el elemento $\{3\}$ pertenece en el conjunto A.
    \item $\{\{3\}\} \subseteq A$: es verdadero ya que todos los elementos dentro del conjunto \{\{3\}\} pertenecen al conjunto A. En este caso, el unico elemento es \{3\} y este pertence a el conjunto A. 
    \item $\{1,2\} \in A$: esto es verdadero ya que el elemento $\{1,2\}$ pertenece al conjunto A.
    \item $\{1,2\} \subseteq A$: es verdadero porque todos los elementos del conjunto $\{1,2\}$ pertenecen al conjunto A. 
    \item $\{\{1,2\}\} \subseteq A$: es verdadero porque todos los elementos del conjunto \{\{1,2\}\} pertenecen al conjunto A. En este caso, $\{1,2\}$ pertenece al conjunto A. 
    \item $\{\{1,2\},3\} \not\subseteq A$: es falso ya que no todos los elementos pertenecen al conjunto. En este caso el elemento $\{1,2\}$ pertenece pero el elemento 3 no pertenece.
    \item $\emptyset \not\subseteq A$: es falso ya que ninguno de los elementos del conjunto A es $\emptyset$.
    \item $\emptyset \subseteq A$: es verdadero ya que el elemento vacío es subconjunto de todos los conjuntos por definición ya que no contiene elementos que pueda contradecir la condición.
    \item $A \not\in A$: es falso ya que dentro de A no existe ningun subconjunto que sea A.
    \item $A \in A$: es verdadero ya que todo conjunto es subconjunto de si por definición.
\end{enumerate}

\section*{3. Determinar si $A \subseteq B$ en cada uno de los siguientes casos.}

\begin{enumerate}[label=\roman*)]
    \item $A = \{1,2,3\}, \quad B = \{5,4,3,2,1\}$
    \item $A = \{1,2,3\}, \quad B = \{1,2,\{3\},-3\}$
    \item $A = \{x \in \mathbb{R} \mid 2 < |x| < 3\}, \quad B = \{x \in \mathbb{R} \mid x^2 < 3\}$
    \item $A = \{\emptyset\}, \quad B = \emptyset$
\end{enumerate}

\textbf{Resolución}
\begin{enumerate}[label=\roman*)]
    \item $A \subseteq B$
    \item $A \not\subseteq B$
    \item $A \not\subseteq B$
    \begin{align*}
        A &= [-3,-2] \cup (2,3) \\
        B &= (-\sqrt{3}, \sqrt{3}) \\[6pt]
        &\text{Entonces, } A \nsubseteq B \\
        &\text{Por ejemplo, } -2.5 \in A \;\text{ pero }\; -2.5 \notin B.
    \end{align*}
    \item $A \not\subseteq B$
\end{enumerate}


\end{document}
